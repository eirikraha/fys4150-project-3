\documentclass{article}
\usepackage{amsmath}
\usepackage[utf8]{inputenc}
\usepackage{graphicx}
\usepackage{verbatim}
\usepackage{float}
\usepackage[makeroom]{cancel}
\usepackage[english]{babel}
\usepackage{textcomp}
\usepackage{gensymb}
\usepackage{color}
\usepackage{subcaption}
\usepackage{caption}
\usepackage{hyperref}
\usepackage{physics}
\usepackage{dsfont}
%\usepackage{amsfonts}
\usepackage{listings}
\usepackage{multicol}
\usepackage{units}

% From Eirik's .tex
\usepackage{epstopdf}
\usepackage{cite}
\usepackage{braket}
\usepackage{url}
\bibliographystyle{plain}

\usepackage{algorithmicx}
\usepackage{algorithm}% http://ctan.org/pkg/algorithms
\usepackage{algpseudocode}% http://ctan.org/pkg/algorithmicx

\usepackage[margin=1cm]{caption}
\usepackage[outer=1.2in,inner=1.2in]{geometry}
% For writing full-size pages
%\usepackage{geometry}
%\geometry{
%  left=5mm,
%  right=5mm,
%  top=5mm,
%  bottom=5mm,
%  heightrounded,
%}

% Finding overfull \hbox
\overfullrule=2cm

\lstset{language=IDL}
 %\lstset{alsolanguage=c++}
\lstset{basicstyle=\ttfamily\small}
 %\lstset{backgroundcolor=\color{white}}
\lstset{frame=single}
\lstset{stringstyle=\ttfamily}
\lstset{keywordstyle=\color{red}\bfseries}
\lstset{commentstyle=\itshape\color{blue}}
\lstset{showspaces=false}
\lstset{showstringspaces=false}
\lstset{showtabs=false}
\lstset{breaklines}
\lstset{aboveskip=20pt,belowskip=20pt}

\lstset{basicstyle=\footnotesize, basewidth=0.5em}
\lstdefinestyle{cl}{frame=none,basicstyle=\ttfamily\small}
\lstdefinestyle{pr}{frame=single,basicstyle=\ttfamily\small}
\lstdefinestyle{prt}{frame=none,basicstyle=\ttfamily\small}
% \lstinputlisting[language=Python]{filename}


\definecolor{codepurple}{rgb}{0.58,0,0.82}
\definecolor{backcolour}{rgb}{0.95,0.95,0.92}
\definecolor{dkgreen}{rgb}{0,0.6,0}
\definecolor{gray}{rgb}{0.5,0.5,0.5}
\definecolor{magenta}{rgb}{0.58,0,0.82}

\lstdefinestyle{pystyle}{
  language=Python,
  aboveskip=3mm,
  belowskip=3mm,
  columns=flexible,
  basicstyle={\small\ttfamily},
  backgroundcolor=\color{backcolour},
  commentstyle=\color{dkgreen},
  keywordstyle=\color{magenta},
  numberstyle=\tiny\color{gray},
  stringstyle=\color{codepurple},
  basicstyle=\footnotesize,
  breakatwhitespace=false,
  breaklines=true,
  captionpos=b,
  keepspaces=true,
  numbers=left,
  numbersep=5pt,
  showspaces=false,
  showstringspaces=false,
  showtabs=false,
  tabsize=2
}

%%%%%%%%%%%%%%%%%%%%%%%%%%%%%%%%
% Self made macros here yaaaaaay
\newcommand\answer[1]{\underline{\underline{#1}}}
\newcommand\pd[2]{\frac{\partial #1}{\partial #2}}
\newcommand\red[1]{\textcolor{red}{\textbf{#1}}}
\newcommand\numberthis{\addtocounter{equation}{1}\tag{\theequation}}
% Usage: \numberthis \label{name}
% Referencing: \eqref{name}

% Some matrices
\newcommand\smat[1]{\big(\begin{smallmatrix}#1\end{smallmatrix}\big)}
\newcommand\ppmat[1]{\begin{pmatrix}#1\end{pmatrix}}

%%%%%%%%%%%%%%%%%%%%%%%%%%%%%%%%%
% Eirik's self made macros
\newcommand{\s}{^{*}}
\newcommand{\V}[1]{\mathbf{#1}}
\newcommand{\husk}[1]{\color{red} #1 \color{black}}
\newcommand{\E}[1]{\cdot 10^{#1}}
\newcommand{\e}[1]{\ \text{#1}}
\newcommand{\tom}[1]{\big( #1 \big)}
\newcommand{\Tom}[1]{\Big( #1 \Big)}
\newcommand{\tomH}[1]{\big[ #1 \big] }
\newcommand{\TomH}[1]{\Big[ #1 \Big]}
\newcommand{\tomK}[1]{ \{ #1 \} }
\newcommand{\TomK}[1]{\Big\lbrace #1 \Big\rbrace}
\newcommand{\bigabs}[1]{\left| #1 \right|}

% Section labeling
\usepackage{titlesec}% http://ctan.org/pkg/titlesec
\renewcommand{\thesubsection}{\arabic{subsection}}


% Title/name/date
\title{FYS4150 - Project 3}
\author{Simen Nyhus Bastnes \& Eirik Ramsli Hauge}
\date{24. October 2016}

\begin{document}
\maketitle
\begin{abstract}
\begin{figure}[H]
\centering
\includegraphics[scale=0.5]{doge.png}
\end{figure}
\end{abstract}
\subsection{Introduction}
In this project we will use Euler's forward method and Velocity Verlet to simulate the solar-system from given initial values. We will start off by looking at a solar system which only contains the Earth and the Sun and then we will expand by including other planets, starting with Jupiter.
\subsection{Theory}
The goal of this project, is to develop a model of our solar system using the so-called velocity Verlet algorithm for solving coupled ordinary differential equations. For our solar system model, the only force interacting on it is gravity.
\subsubsection{The Earth-Sun system}
To start off, we simplify by looking at a hypothetical solar system with only the Earth orbiting around the Sun, where any solar motion is small enough to be neglected. Newton's law of gravitation, gives us then
\begin{align*}
  F_G &= \frac{GM_{\odot}M_{\text{Earth}}}{r^2}
\intertext{where $M_{\odot}$ is the solar mass, $M_{\text{Earth}}$ is the mass of the Earth, $G$ the gravitational constant, and $r$ is the distance between the Earth and the Sun. Written on component form, $F_G$ takes the following form}
  F_{G,\mathbf{r}} &= \frac{GM_{\odot}M_{\text{Earth}}}{r^3}\,\mathbf{r}
\end{align*}
 where $\mathbf{r} = x\mathbf{i}+y\mathbf{j}+z\mathbf{k}$ is the vector between the Sun and the Earth.
\begin{align*}
  F_{G,\mathbf{r}} &= \frac{GM_{\odot}M_{\text{Earth}}}{r^3}\,(x\mathbf{i}+y\mathbf{j}+z\mathbf{k})\numberthis\label{eq:FG_comp}
\end{align*}
\\Using Newton's second law of motion, we can then get the following differential equations for the motion of the Earth.
\begin{align*}
  \frac{d^2x}{dt^2} = \frac{F_{G,x}}{M_{Earth}}\numberthis\label{eq:diffx}\\
  \frac{d^2y}{dt^2} = \frac{F_{G,y}}{M_{Earth}}\numberthis\label{eq:diffy}\\
  \frac{d^2z}{dt^2} = \frac{F_{G,z}}{M_{Earth}}\numberthis\label{eq:diffz}
\end{align*}
where $F_{G,x}$, $F_{G,y}$, and $F_{G,z}$ are the components of the gravitational force, given by \eqref{eq:FG_comp}. We can assume that the orbit of the Earth (and other objects in the solar system) around the Sun is mostly co-planar, so we could take this to be the $xy$-plane, reducing our differential equations to equations \eqref{eq:diffx} and \eqref{eq:diffy}. However, since looking at the equations in three dimensions isn't that much more work, we will continue doing so.\\\\
We can obtain mass units from using that Earth's orbit is almost circular around the Sun. For circular motion, we know that the force obeys the following relation
\begin{align*}
F_G &= \frac{M_{\text{Earth}}v^2}{r} = \frac{GM_{\odot}M_{\text{Earth}}}{r^2}
\end{align*}
where $v$ is the velocity of the Earth. This can be shown to give us the useful relation
\begin{align*}
  v^2r = GM_{\odot} = 4\pi^2\text{AU}^3/\text{yr}^2\numberthis\label{eq:mass_units}
\end{align*}
which we can use to scale both distance and time units to more ideal orders of magnitudes. One astronomical unit (1 AU) is defined as the average distance from the Earth to the Sun, or roughly $10^{11}$ m. Inserting this into equation \eqref{eq:FG_comp}, gives us
\begin{align*}
  F_{G,\mathbf{r}} &= \frac{4\pi^2\text{AU}^3/\text{yr}^2\,M_{\text{Earth}}}{r^3}(x\mathbf{i}+y\mathbf{j}+z\mathbf{k})
\end{align*} 
where $r$ now is given in units AU. \red{not fully satisfied here, try to add some more about how scaling is nice, or just some other shit.}
\\\\
\red{might want to discretize in the section about numerically solving, or just do it here?}In order to solve the differential equations given by equations \eqref{eq:diffx}, \eqref{eq:diffy} and \eqref{eq:diffz} numerically, we need to discretize them. For simplicity, we look at only the $x$ direction, though exactly the same can be done for the $y$ and $z$ directions. We define $x$ in the interval $x_{\text{min}} \dots x_{\text{max}}$. We define the step length $h$, so that we can discretize $x$ as $x_i$, where\red{meh}
\subsubsection{The three-body problem}
\red{adding jupyter}
\subsubsection{Potential energy/circular orbit/escape velocity?}

\subsubsection{Numerically solving ordinary differential equations}
Euler's forward method \husk{REF!} is a well known numerical method used to solve differential equations. The method is fairly simple, basing itself on that the next step is the former step added with its own derivative times a steplength. By defining $h$ as the step length we can write express Euler's forward method as follows:
\begin{equation}
x_{i + 1} = x_i + x^{'}_i \cdot h
\label{eq:fwdEuler}
\end{equation}
where we define $x_i$ as $x(i)$, $x_{i+1} = x(i + h)$ and $i = 0, 1, 2 ... , n$. \\
The Verlet Velocity method presents a different approach to solve the differential equations in this project. While Euler's method is applicable to many differential equations, the Verlet methods are specified to solve Newton's equations of motion \husk{REF}. \husk{Write the rest of the explanation. The method is written down both on Wiki and in the kladdebok on the second whole page from 30.09.2016}
\subsection{Experimental}
\subsection{Results}
\subsection{Discussion}
\subsection{Conclusion}
\end{document}
