\documentclass{article}
\usepackage{amsmath}
\usepackage[utf8]{inputenc}
\usepackage{graphicx}
\usepackage{verbatim}
\usepackage{float}
\usepackage[makeroom]{cancel}
\usepackage[english]{babel}
\usepackage{textcomp}
\usepackage{gensymb}
\usepackage{color}
\usepackage{subcaption}
\usepackage{caption}
\usepackage{hyperref}
\usepackage{physics}
\usepackage{dsfont}
%\usepackage{amsfonts}
\usepackage{listings}
\usepackage{multicol}
\usepackage{units}

% From Eirik's .tex
\usepackage{epstopdf}
\usepackage{cite}
\usepackage{braket}
\usepackage{url}
\bibliographystyle{plain}

\usepackage{algorithmicx}
\usepackage{algorithm}% http://ctan.org/pkg/algorithms
\usepackage{algpseudocode}% http://ctan.org/pkg/algorithmicx

\usepackage[margin=1cm]{caption}
\usepackage[outer=1.2in,inner=1.2in]{geometry}
% For writing full-size pages
%\usepackage{geometry}
%\geometry{
%  left=5mm,
%  right=5mm,
%  top=5mm,
%  bottom=5mm,
%  heightrounded,
%}

% Finding overfull \hbox
\overfullrule=2cm

\lstset{language=IDL}
 %\lstset{alsolanguage=c++}
\lstset{basicstyle=\ttfamily\small}
 %\lstset{backgroundcolor=\color{white}}
\lstset{frame=single}
\lstset{stringstyle=\ttfamily}
\lstset{keywordstyle=\color{red}\bfseries}
\lstset{commentstyle=\itshape\color{blue}}
\lstset{showspaces=false}
\lstset{showstringspaces=false}
\lstset{showtabs=false}
\lstset{breaklines}
\lstset{aboveskip=20pt,belowskip=20pt}

\lstset{basicstyle=\footnotesize, basewidth=0.5em}
\lstdefinestyle{cl}{frame=none,basicstyle=\ttfamily\small}
\lstdefinestyle{pr}{frame=single,basicstyle=\ttfamily\small}
\lstdefinestyle{prt}{frame=none,basicstyle=\ttfamily\small}
% \lstinputlisting[language=Python]{filename}


\definecolor{codepurple}{rgb}{0.58,0,0.82}
\definecolor{backcolour}{rgb}{0.95,0.95,0.92}
\definecolor{dkgreen}{rgb}{0,0.6,0}
\definecolor{gray}{rgb}{0.5,0.5,0.5}
\definecolor{magenta}{rgb}{0.58,0,0.82}

\lstdefinestyle{pystyle}{
  language=Python,
  aboveskip=3mm,
  belowskip=3mm,
  columns=flexible,
  basicstyle={\small\ttfamily},
  backgroundcolor=\color{backcolour},
  commentstyle=\color{dkgreen},
  keywordstyle=\color{magenta},
  numberstyle=\tiny\color{gray},
  stringstyle=\color{codepurple},
  basicstyle=\footnotesize,
  breakatwhitespace=false,
  breaklines=true,
  captionpos=b,
  keepspaces=true,
  numbers=left,
  numbersep=5pt,
  showspaces=false,
  showstringspaces=false,
  showtabs=false,
  tabsize=2
}

%%%%%%%%%%%%%%%%%%%%%%%%%%%%%%%%
% Self made macros here yaaaaaay
\newcommand\answer[1]{\underline{\underline{#1}}}
\newcommand\pd[2]{\frac{\partial #1}{\partial #2}}
\newcommand\red[1]{\textcolor{red}{\textbf{#1}}}
\newcommand\numberthis{\addtocounter{equation}{1}\tag{\theequation}}
% Usage: \numberthis \label{name}
% Referencing: \eqref{name}

% Some matrices
\newcommand\smat[1]{\big(\begin{smallmatrix}#1\end{smallmatrix}\big)}
\newcommand\ppmat[1]{\begin{pmatrix}#1\end{pmatrix}}

%%%%%%%%%%%%%%%%%%%%%%%%%%%%%%%%%
% Eirik's self made macros
\newcommand{\s}{^{*}}
\newcommand{\V}[1]{\mathbf{#1}}
\newcommand{\husk}[1]{\color{red} #1 \color{black}}
\newcommand{\E}[1]{\cdot 10^{#1}}
\newcommand{\e}[1]{\ \text{#1}}
\newcommand{\tom}[1]{\big( #1 \big)}
\newcommand{\Tom}[1]{\Big( #1 \Big)}
\newcommand{\tomH}[1]{\big[ #1 \big] }
\newcommand{\TomH}[1]{\Big[ #1 \Big]}
\newcommand{\tomK}[1]{ \{ #1 \} }
\newcommand{\TomK}[1]{\Big\lbrace #1 \Big\rbrace}
\newcommand{\bigabs}[1]{\left| #1 \right|}

% Section labeling
\usepackage{titlesec}% http://ctan.org/pkg/titlesec
\renewcommand{\thesubsection}{\arabic{subsection}}


% Title/name/date
\title{FYS4150 - Project 3}
\author{Simen Nyhus Bastnes \& Eirik Ramsli Hauge}
\date{24. October 2016}

\begin{document}
\maketitle
\begin{abstract}
In this project we will use Euler's forward method and Velocity Verlet to simulate the solar-system from given initial values. We will start off by looking at a solar system which only contains the Earth and the Sun and then we will expand by including other planets, starting with Jupiter. We will also include tests for conservation of kinetic energy, potential energy and angular momentum and find the escape velocity of an Earth like planet in a single planet solar system. Lastly we will implement a general relativity correction to the gravity force and compare the perihelion angle with that given from classical Newtonian gravitation. We found that \husk{Write results summary}
\end{abstract}
\subsection{Introduction}
Differential equations are a big part of physics and one of the most known differential eqution can be said to be Newtons second law of motion. Many of our known differential equations can not be solved analytically for more than a few special cases. However, with the power of computers, we have been able to solve increasingly complex differential equations with no analytical solution numerically. To be able to solve differential equations numerically is therefore a great tool for any physicist and this project is designed to give us a basic understandig of how to solve a differential equation and as a bonus, how to use object orientation in C++. \\ \\
Our aim with this project is to simulate an entire solar system with the two methods forward Euler and Velocity Verlet. We will start with a system only containing the Sun and the Earth to compare our two methods. Afterwards we will run a test to see if the values for kinetic energy, potential energy and angular momentum are conserved. Thirdly we will see if we can find the escape velcoity of the Earth with trial and error and check if the velocity coincides with our analytical escape velocity. \\
The Sun-Earth system will be the basis for the rest of our solar system. When we are sure the smaller system works, we will add more planets (and Pluto) starting with Jupiter. Using object orientation and the fact that only the gravitational force is acting between the celestial bodies, we will find the sum of all forces working on a celestial body from the other celestial bodies. By accessing initail conditions from NASA we will relatively accurately simulate the whole solar system, which is pretty cool. \\
In the final part of the project we will add a relativistic correction to the gravitational force and compare the perhelion angle between the classic and the relativstic corrected case for a Sun-Mercury system. If the difference is about 43 arcseconds after a century, we will have affirmed Einsteins theory of general relativity. Which is also pretty cool.
\subsection{Theory}
The goal of this project, is to develop a model of our solar system using the so-called velocity Verlet algorithm for solving coupled ordinary differential equations. For our solar system model, the only force interacting on it is gravity.
\subsubsection{The Earth-Sun system}
To start off, we simplify by looking at a hypothetical solar system with only the Earth orbiting around the Sun, where any solar motion is small enough to be neglected. Newton's law of gravitation, gives us then
\begin{align*}
  F_G &= \frac{GM_{\odot}M_{\text{Earth}}}{r^2}
\intertext{where $M_{\odot}$ is the solar mass, $M_{\text{Earth}}$ is the mass of the Earth, $G$ the gravitational constant, and $r$ is the distance between the Earth and the Sun. Written on component form, $F_G$ takes the following form}
  F_{G,\mathbf{r}} &= \frac{GM_{\odot}M_{\text{Earth}}}{r^3}\,\mathbf{r}
\end{align*}
 where $\mathbf{r} = x\mathbf{i}+y\mathbf{j}+z\mathbf{k}$ is the vector between the Sun and the Earth.
\begin{align*}
  F_{G,\mathbf{r}} &= \frac{GM_{\odot}M_{\text{Earth}}}{r^3}\,(x\mathbf{i}+y\mathbf{j}+z\mathbf{k})\numberthis\label{eq:FG_comp}
\end{align*}
\\Using Newton's second law of motion, we can then get the following differential equations for the motion of the Earth.
\begin{align*}
  \frac{d^2x}{dt^2} = \frac{F_{G,x}}{M_{Earth}}\numberthis\label{eq:diffx}\\
  \frac{d^2y}{dt^2} = \frac{F_{G,y}}{M_{Earth}}\numberthis\label{eq:diffy}\\
  \frac{d^2z}{dt^2} = \frac{F_{G,z}}{M_{Earth}}\numberthis\label{eq:diffz}
\end{align*}
where $F_{G,x}$, $F_{G,y}$, and $F_{G,z}$ are the components of the gravitational force, given by \eqref{eq:FG_comp}. We can assume that the orbit of the Earth (and other objects in the solar system) around the Sun is mostly co-planar, so we could take this to be the $xy$-plane, reducing our differential equations to equations \eqref{eq:diffx} and \eqref{eq:diffy}. However, since looking at the equations in three dimensions isn't that much more work, we will continue doing so.\\\\
We can obtain mass units from using that Earth's orbit is almost circular around the Sun. For circular motion, we know that the force obeys the following relation
\begin{align*}
F_G &= \frac{M_{\text{Earth}}v^2}{r} = \frac{GM_{\odot}M_{\text{Earth}}}{r^2}
\end{align*}
where $v$ is the velocity of the Earth. This can be shown to give us the useful relation
\begin{align*}
  v^2r = GM_{\odot} = 4\pi^2\text{AU}^3/\text{yr}^2\numberthis\label{eq:mass_units}
\end{align*}
which we can use to scale both distance and time units to more ideal orders of magnitudes. One astronomical unit (1 AU) is defined as the average distance from the Earth to the Sun, or roughly $10^{11}$ m. Inserting this into equation \eqref{eq:FG_comp}, gives us
\begin{align*}
  F_{G,\mathbf{r}} &= \frac{4\pi^2\text{AU}^3/\text{yr}^2\,M_{\text{Earth}}}{r^3}(x\mathbf{i}+y\mathbf{j}+z\mathbf{k})
\end{align*} 
where $r$ now is given in units AU. \red{not fully satisfied here, try to add some more about how scaling is nice, or just some other shit.}
\\\\
\red{might want to discretize in the section about numerically solving, or just do it here?}In order to solve the differential equations given by equations \eqref{eq:diffx}, \eqref{eq:diffy} and \eqref{eq:diffz} numerically, we need to discretize them. For simplicity, we look at only the $x$ direction, though exactly the same can be done for the $y$ and $z$ directions. We define $x$ in the interval $x_{\text{min}} \dots x_{\text{max}}$. We define the step length $h$, so that we can discretize $x$ as $x_i$, where\red{meh}
\subsubsection{The three-body problem}
\red{adding jupyter}
\subsubsection{Potential energy/circular orbit/escape velocity?}

\subsubsection{Numerically solving ordinary differential equations}
Euler's forward method \husk{REF!} is a well known numerical method used to solve differential equations. The method is fairly simple, basing itself on that the next step is the former step added with its own derivative times a steplength. By defining $h$ as the step length we can write express Euler's forward method as follows:
\begin{equation}
f_{i + 1} = f_i + f^{'}_i \cdot h
\label{eq:fwdEuler}
\end{equation}
where we define $f_i$ as $f(i)$, $f_{i+1} = f(i + h)$ and $i = 0, 1, 2 ... , n$. \\
In our case we use Newton's laws and we want to find the position and velocity of our celestial bodies. Since it's easy to find the acceleration from equation \eqref{eq:FG_comp}, and we know that $a = \frac{\partial^2 x}{\partial t^2}$ and $v = \frac{\partial x}{\partial t}$ where $x$ is position, $v$ is velocity and $a$ is acceleration, Euler's forward method becomes:
\begin{equation}
v_{i + 1} = v_i + h \cdot a_i + \mathcal{O}(h^2)\\
\label{eq:Eulervel}
\end{equation}
\husk{Uncertain about the error, check before delivery}
\begin{equation}
x_{i + 1} = x_i + h \cdot v_i + \mathcal{O}(h^3)\\
\label{eq:Eulerpos}
\end{equation}
The Verlet Velocity method presents a different approach to solve the differential equations in this project. While Euler's method is applicable to many differential equations, the Verlet methods are specified to solve Newton's equations of motion \husk{REF}. The Verlet Velocity method is derived by first Taylor expanding $x_{i + 1}$ ad $x_{i-1}$:
\begin{align*}
x_{i-1} &\simeq x_i - v_i h + \frac{1}{2}a_i h^2 \\
x_{i+1} &\simeq x_i + v_i h + \frac{1}{2}a_i h^2
\intertext{By adding the two expressions we can find an expression for $x_{i+1}$}
x_{i+1} + x_{i-1} &= 2x_i + a_i h^2 \\
x_{i+1} &= 2x_i - x_{i-1} + a_i h^2 \\
\intertext{and by subtracting them we can find an expression for $v_{i}$}
x_{i+1} - x_{i-1} &= 2h v_i \\
v_i &= \frac{x_{i+1} - x_{i-1}}{2h}
\intertext{These expressions for $x_{i+1}$ and $v_i$ are called the Verlet method. However, we want to have an expression for $v_{i+1}$.}
v_{i+1} &= \frac{x_{i+2} - x_i}{2h} \\
\intertext{We can easily find $x_{i+1}$ by using the expression above}
x_{i+2} &= 2x_{i+1} - x_i + a_{i+1} h^2 \\
\intertext{Inserting this back into $v_{i+1}$ gives us:}
v_{i+1} &= \frac{2x_{i+1} - x_i + a_{i+1} h^2 - x_i}{2h} \\
v_{i+1} &= \frac{x_{i+1} - x_i}{h} + \frac{1}{2} a_{i+1} h
\intertext{Inserting the Taylor expanded $x_{i+1}$ for $v_i$ dependency}
v_{i+1} &= \frac{x_i + v_i h + \frac{1}{2} a_i h^2 - x_i}{h} + \frac{1}{2} a_{i+1} h \\
v_{i+1} &= v_i + \frac{1}{2} a_i h + \frac{1}{2} a_{i+1} h \\
\end{align*}
This gives us the equations for position and velocity in the velocity Verlet method:
\begin{equation}
x_{i+1} = x_i + v_i h + \frac{1}{2} a_i h^2
\label{eq:Verletpos}
\end{equation}
\begin{equation}
v_{i+1} = v_i + \frac{1}{2} h (a_i + a_{i+1})
\label{eq:Verletvel}
\end{equation}
As we can see, the term for $v_{i+1}$ is dependent on $a_{i+1}$. Therefore the algorithm must go like this:
\begin{algorithm}[H]
\small
\caption{Velocity Verlet}\label{alg:VelVerlet}
\begin{algorithmic}[1]
\For{$i = 0, n$}
\State $a_i = F_{G, r_i}/m$
\State $r_i = r_i + v_i h + \frac{1}{2} a_i h^2$
\State $a_{i+1} = F_{G, r_{i+1}}/m$
\State $v_{i+1} = v_i + \frac{1}{2} h (a_i + a_{i+1})$
\EndFor
\end{algorithmic}
\end{algorithm}
Where we calculate $F_{G, r}$ by euation \eqref{eq:FG_comp}. \husk{Should we change to the one we actually use?}
\subsubsection*{Perihelion precission}
Lastly we want to study the perihelion angle of Mercury. This was an important test of the general relativety. When one compared the observed value of the perhelion procession to all classical effects, there was a difference of about 43 arcseconds after a century. This difference was not understood until Einstein predicted it with general relativity. Einstein found that the perihelion precission could be predicted by:
\begin{equation}
\epsilon = 24 \pi^3 \frac{\alpha^2}{T^2 c^2 (1 - e^2)}
\label{eq:perihelion}
\end{equation}
Where $\alpha$ is the major semi-axis of the planets orbit, $e$ is the eccentricity of the orbit, $T$ the period of revolution and $c$ the speed of light. By inserting the right values for Mercury, it was found that for a year the precission was about 0.104 arcseconds and after a (Earth) century, Mercury would have revolved about 415 times around the sun giving a precission of about 43 arcseconds which coincides with the difference between the observed and classically calculated value thus proving general relativity. \\
It became clear that the Newtonian force of gravity needed a relativistic correction. In our numerical calculations, we will therefore add a correction setting the gravitational force to:
\begin{equation}
F_G = \frac{G M_{\odot} M_{Mercury}}{r^2} \, \, \TomH{1 + \frac{3l^2}{r^2c^2}}
\label{eq:FgGR}
\end{equation}
Where $l$ is the orbital angular momentum given as $l = |\vec{r} \cross \vec{v}|$ for Mercury and $c$ is once again the speed of light.
We can then find the perihelion angle by
\begin{equation}
\tan \theta_p = \frac{y_p}{x_p}
\end{equation}
where $y_p$ and $x_p$ are $y$ and $x$ respectively at perihelion.
\subsection{Experimental}
\subsection{Results}
\subsubsection*{Perihelion precission}
By using Verlet twice, first with acceleration given by equation \eqref{eq:FG_comp} and then by the component version of equation \eqref{eq:FgGR}, we can compare the perihelion angles. By setting the velocity to 12.44 AU/year, and the initial position to be (0.3075, 0, 0) AU we found \husk{Fyll in her med figur}
\subsection{Discussion}
\subsection{Conclusion}
\end{document}
